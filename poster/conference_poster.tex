%==============================================================================
% Sjabloon poster bachproef
%==============================================================================
% Gebaseerd op document class `a0poster' door Gerlinde Kettl en Matthias Weiser
% Aangepast voor gebruik aan HOGENT door Jens Buysse en Bert Van Vreckem

\documentclass[a0,portrait]{hogent-poster}

% Info over de opleiding
\course{Bachelorproef}
\studyprogramme{toegepaste informatica}
\academicyear{2022-2023}
\institution{Hogeschool Gent, Valentin Vaerwyckweg 1, 9000 Gent}

% Info over de bachelorproef
\title{Alternatieven voor Citrix Podio voor het ontwikkelen van low-code business management software.}
\subtitle{Vergelijkende Studie}
\author{Cédrick Lauwyck}
\email{cedrick.lauwyck@student.hogent.be}
\supervisor{Lena De Mol}
\cosupervisor{Bavo De Cooman (Quivvy Solutions BV)}

% Indien ingevuld, wordt deze informatie toegevoegd aan het einde van de
% abstract. Zet in commentaar als je dit niet wilt.
\specialisation{Mobile \& Enterprise Developer}
\keywords{Low-code, Podio, Airtable, Google Appsheet}
% \projectrepo{https://github.com/user/repo}

\begin{document}

\maketitle

\begin{abstract}
    
Dit onderzoek werd gedaan op aanvraag van bedrijf Quivvy Solutions BV en is ook specifiek naar hen toe gericht. Zij werken al lange tijd met Podio, een low-code cloud-based management platform, maar toch merken ze de laatste jaren steeds meer stabiliteitsproblemen op. Om die redenen houden ze de optie open om op lange termijn over te stappen naar een alternatief platform. Dit platform moet uiteraard over dezelfde functionaliteiten als Podio beschikken en het moet vervolgens beter presteren op gebied van stabiliteit, automatisaties, relaties, calculaties, $\ldots$ In deze studie wordt een vergelijking gemaakt tussen Podio en twee alternatieven, namelijk Airtable en Google AppSheet. Dit om een antwoord te bekomen op de vraag: 'Met welke tools kan Quivvy Solutions hetzelfde bereiken als met Podio en hoe geïntegreerd werkt dit?'. \\

Eerst en vooral wordt in de stand van zaken toegelicht wat er precies verstaan wordt onder de term 'low-code/no-code', welke verschillen er zijn tussen de twee en welke uitdagingen ze met zich meebrengen. Het zijn namelijk platformen die een gebruiker toestaan om applicaties en software te ontwikkelen zonder dat enige programmeerkennis vereist is. Verder wordt er uitgelegd waarom zoveel bedrijven voor low-code/no-code platformen kiezen. Vervolgens wordt wat dieper ingegaan op Podio zelf en de twee potentiële alternatieven, namelijk Airtable en Google Appsheet. Er wordt uitgelegd wat ze precies zijn, hoe ze te werk gaan en wat ze allemaal te bieden hebben. \\

Daarna komt de methodologie en het onderzoek aan bod. Er wordt eerst een vooronderzoek met requirementsanalyse uitgevoerd. Hieruit worden vervolgens twee potentiële alternatieven, Airtable en Google AppSheet, gehaald en verder worden deze keuzes ondersteund. Dit wordt gevolgd door het effectieve onderzoek, waarin een use case wordt uitgebouwd in alle voorgenoemde platformen en daarna worden hun functionaliteiten vergeleken met Podio. Deze use case bestaat uit het bijhouden van een stageproces, namelijk het aanmaken van een studentendossier met evaluaties en sollicitaties voor een student. \\ 

Ten slotte wordt een conclusie gemaakt, waarin besloten wordt dat zowel Airtable als AppSheet sterke alternatieven zijn voor Podio. Vervolgens wordt er ook besloten dat Airtable het best presteert op gebied van gebruikerservaring en dat Podio nog steeds een koploper is op gebied van calculaties. Dit onderzoek geeft ook aanzet tot een vervolgstudie waarin twee andere alternatieven kunnen worden vergeleken met Podio. \\

\end{abstract}

\begin{multicols}{2} % This is how many columns your poster will be broken into, a portrait poster is generally split into 2 columns

\section{Introductie}

De afgelopen jaren hebben bedrijven steeds meer nood aan een digitale gecentraliseerde structuur waarin al hun processen en gegevens samenkomen. Hieruit kunnen ze hun processen beter opvolgen en vervolgens betere business beslissingen maken. Om die reden werd Podio \footnote{https://www.podio.com/} ontwikkeld, een low-code tool waarmee dit soort cloud based business management platform opgebouwd kan worden. In de laatste jaren is de low-code/no-code markt enorm gegroeid, tegen 2024 zouden zelfs meer dan 65\% van de applicaties gemaakt worden in een low-code/no-code omgeving. Tegenwoordig is Podio echter niet meer het enige platform dat hiertoe instaat is, maar zijn er al een groot aantal concurrenten of alternatieven op de markt gekomen. Hieruit volgt nu de vraag of Podio nog steeds het meest optimale platform is, of zijn er eventueel platformen die beter presteren? \\

\section{Onderzoek}

TODO

% \begin{center}
%   \captionsetup{type=figure}
%   \includegraphics[width=1.0\linewidth]{grail}
%   \captionof{figure}{He hasn't got shit all over him. The nose? Where'd you get the coconuts? What do you mean? We shall say `Ni' again to you, if you do not appease us}
% \end{center}

Let er wel op dat dit tot problemen met bladschikking kan leiden.

\section{Conclusies}

Na het onderzoek wordt geconcludeerd dat AppSheet en Airtable een betere implementatie hebben voor het leggen van relaties tussen onderlinge gegevens. Daarnaast blijft Podio nog steeds veruit het beste platform op gebied van calculaties, omdat het gebruikers namelijk veel vrijheid geeft om complexe berekeningen en tekstverwerkingen te maken. Verder blinkt Airtable uit op gebied van performantie en gebruikerservaring. Het platform vereist het minste aantal klikken en de pagina's worden zeer snel ingeladen, waardoor gebruikers efficiënt kunnen werken zonder tijd te verspillen aan wachten. Wat betreft automatisaties kunnen de platformen als min of meer gelijkwaardig beschouwd worden, omdat ze bij elk platform zowel voordelen als nadelen hebben ten opzichte van elkaar. \\

Ten slotte kan er besloten worden dat Quivvy Solutions zowel met Airtable als met Google AppSheet dezelfde doeleinden kan bereiken als met Podio, maar dat er wel een nodige inspanning moet gedaan worden om te leren werken met beide alternatieven Daarnaast kan uit de resultaten van het onderzoek afgeleid worden dat uit de twee alternatieven, Airtable waarschijnlijk een betere keuze is voor het bedrijf dan AppSheet, aangezien het bouwproces probleemloos verliep en het platform de beste gebruikerservaring heeft. \\

\section{Toekomstig onderzoek}

Aangezien er steeds meer low-code platformen uitkomen op de markt is er zeker plaats voor een vervolgonderzoek waar Podio vergeleken kan worden met andere mogelijke alternatieven zoals Notion of Mendix. Daarnaast kan er ook dieper ingegaan worden op specifieke extensies of functionaliteiten binnen Podio. \\ 

\end{multicols}
\end{document}