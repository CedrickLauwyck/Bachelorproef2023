%%=============================================================================
%% Samenvatting
%%=============================================================================

% TODO: De "abstract" of samenvatting is een kernachtige (~ 1 blz. voor een
% thesis) synthese van het document.
%
% Een goede abstract biedt een kernachtig antwoord op volgende vragen:
%
% 1. Waarover gaat de bachelorproef?
% 2. Waarom heb je er over geschreven?
% 3. Hoe heb je het onderzoek uitgevoerd?
% 4. Wat waren de resultaten? Wat blijkt uit je onderzoek?
% 5. Wat betekenen je resultaten? Wat is de relevantie voor het werkveld?
%
%
% LET OP! Een samenvatting is GEEN voorwoord!

%%---------- Nederlandse samenvatting -----------------------------------------
%
% TODO: Als je je bachelorproef in het Engels schrijft, moet je eerst een
% Nederlandse samenvatting invoegen. Haal daarvoor onderstaande code uit
% commentaar.
% Wie zijn bachelorproef in het Nederlands schrijft, kan dit negeren, de inhoud
% wordt niet in het document ingevoegd.

\IfLanguageName{english}{%
\selectlanguage{dutch}
\chapter*{Samenvatting}
\lipsum[1-4]
\selectlanguage{english}
}{}

%%---------- Samenvatting -----------------------------------------------------
% De samenvatting in de hoofdtaal van het document

% Daarom bestaat een abstract uit volgende componenten:
%
% - inleiding + kaderen thema
% - probleemstelling
% - (centrale) onderzoeksvraag
% - onderzoeksdoelstelling
% - methodologie
% - resultaten (beperk tot de belangrijkste, relevant voor de onderzoeksvraag)
% - conclusies, aanbevelingen, beperkingen

\chapter*{\IfLanguageName{dutch}{Samenvatting}{Abstract}}

Dit onderzoek werd gedaan op aanvraag van bedrijf Quivvy Solutions BV en is ook specifiek naar hen toe gericht. Zij werken al lange tijd met Podio, een low-code cloud-based management platform, maar toch merken ze de laatste jaren steeds meer stabiliteitsproblemen op. Om die redenen houden ze de optie open om op lange termijn over te stappen naar een alternatief platform. Dit platform moet uiteraard over dezelfde functionaliteiten als Podio beschikken en het moet vervolgens beter presteren op gebied van stabiliteit, automatisaties, relaties, calculaties, $\ldots$ In deze studie wordt een vergelijking gemaakt tussen Podio en twee alternatieven, namelijk Airtable en Google AppSheet. Dit om een antwoord te bekomen op de vraag: 'Met welke tools kan Quivvy Solutions hetzelfde bereiken als met Podio en hoe geïntegreerd werkt dit?'. \\

Eerst en vooral wordt in de stand van zaken toegelicht wat er precies verstaan wordt onder de term 'low-code/no-code', welke verschillen er zijn tussen de twee en welke uitdagingen ze met zich meebrengen. Het zijn namelijk platformen die een gebruiker toestaan om applicaties en software te ontwikkelen zonder dat enige programmeerkennis vereist is. Verder wordt er uitgelegd waarom zoveel bedrijven voor low-code/no-code platformen kiezen. Vervolgens wordt wat dieper ingegaan op Podio zelf en de twee potentiële alternatieven, namelijk Airtable en Google Appsheet. Er wordt uitgelegd wat ze precies zijn, hoe ze te werk gaan en wat ze allemaal te bieden hebben. \\

Daarna komt de methodologie en het onderzoek aan bod. Er wordt eerst een vooronderzoek met requirementsanalyse uitgevoerd. Hieruit worden vervolgens twee potentiële alternatieven, Airtable en Google AppSheet, gehaald en verder wordt deze keuzes ondersteund. Dit wordt gevolgd door het effectieve onderzoek, waarin een use case wordt uitgebouwd in alle voorgenoemde platformen en daarna worden hun functionaliteiten vergeleken met Podio. Deze use case bestaat uit het bijhouden van een stageproces, namelijk het aanmaken van een studentendossier met evaluaties en sollicitaties voor een student. \\ 

Ten slotte wordt een conclusie gemaakt, waarin besloten wordt dat zowel Airtable als AppSheet sterke alternatieven zijn voor Podio. Vervolgens wordt er ook besloten dat Airtable het best presteert op gebied van gebruikerservaring en dat Podio nog steeds een koploper is op gebied van calculaties. Dit onderzoek geeft ook aanzet tot een vervolgstudie waarin 2 andere alternatieven kunnen worden vergeleken met Podio. \\

