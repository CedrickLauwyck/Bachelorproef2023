%%=============================================================================
%% Inleiding
%%=============================================================================

\chapter{\IfLanguageName{dutch}{Inleiding}{Introduction}}% 
\label{ch:inleiding} % TODO - OK

De afgelopen jaren hebben bedrijven steeds meer nood aan een digitale gecentraliseerde structuur waarin al hun processen en gegevens samenkomen. Hieruit kunnen ze hun processen beter opvolgen en vervolgens betere business beslissingen maken. Om die reden werd Podio \footnote{https://www.podio.com/} ontwikkeld, een low-code tool waarmee dit soort cloud based business management platform opgebouwd kan worden. In de laatste jaren is de low-code/no-code markt enorm gegroeid. Tegen 2024 zouden zelfs meer dan 65\% van de applicaties gemaakt worden in een low-code/no-code omgeving. Tegenwoordig is Podio echter lang niet meer het enige platform die hiertoe instaat is, maar zijn er al een groot aantal concurrenten of alternatieven op de markt gekomen. Hieruit volgt nu de vraag of Podio nog steeds het meest optimale platform is, of zijn er eventueel platformen die beter presteren? 

\section{\IfLanguageName{dutch}{Probleemstelling}{Problem Statement}}%
\label{sec:probleemstelling} % TODO - OK

% Uit je probleemstelling moet duidelijk zijn dat je onderzoek een meerwaarde heeft voor een concrete doelgroep. De doelgroep moet goed gedefinieerd en afgelijnd zijn. Doelgroepen als ``bedrijven,'' ``KMO's'', systeembeheerders, enz.~zijn nog te vaag. Als je een lijstje kan maken van de personen/organisaties die een meerwaarde zullen vinden in deze bachelorproef (dit is eigenlijk je steekproefkader), dan is dat een indicatie dat de doelgroep goed gedefinieerd is. Dit kan een enkel bedrijf zijn of zelfs één persoon (je co-promotor/opdrachtgever).

Dit onderzoek werd gedaan voor bedrijven die specialiseren in het gebruik van Podio, maar voornamelijk voor Quivvy Solutions BV\footnote{https://quivvy.com}. Het bedrijf maakt low-code software oplossing op maat, met een minimum tools en een maximum en functionaliteit. Hiervoor gebruiken zij Podio, alhoewel de tool voldoet aan hun eisen, hebben ze de laatste jaren al een aantal stabiliteitsproblemen ondervonden. Daarnaast zijn er ook al een groot aantal concurrerende tools op de markt en houden ze dus de optie open om op termijn over te schakelen naar een goed alternatief. 

\section{\IfLanguageName{dutch}{Onderzoeksvraag}{Research question}}%
\label{sec:onderzoeksvraag} % TODO - OK

% Wees zo concreet mogelijk bij het formuleren van je onderzoeksvraag. Een onderzoeksvraag is trouwens iets waar nog niemand op dit moment een antwoord heeft (voor zover je kan nagaan). Het opzoeken van bestaande informatie (bv. ``welke tools bestaan er voor deze toepassing?'') is dus geen onderzoeksvraag. Je kan de onderzoeksvraag verder specifiëren in deelvragen. Bv.~als je onderzoek gaat over performantiemetingen, dan 

De onderzoeksvraag voor deze studie gaat als volgt: Met welke tools kan Quivvy Solutions hetzelfde bereiken als met Podio en hoe geïntegreerd werkt dit?

Deze vraag kan nog verder opgedeeld worden in volgende deelvragen:
\begin{itemize}
    \item Welke verschillen zijn er tussen Podio en de onderzochte alternatieven?
    \item Op welke gebieden presteren de alternatieven beter of slechter dan Podio?
    \item Is de overstap van Podio naar een nieuwer alternatief de nodige inspanning waard?
\end{itemize}

\section{\IfLanguageName{dutch}{Onderzoeksdoelstelling}{Research objective}}%
\label{sec:onderzoeksdoelstelling} % TODO - OK

% Wat is het beoogde resultaat van je bachelorproef? Wat zijn de criteria voor succes? Beschrijf die zo concreet mogelijk. Gaat het bv.\ om een proof-of-concept, een prototype, een verslag met aanbevelingen, een vergelijkende studie, enz.
Het beoogde resulaat van deze vergelijkende studie is om te kunnen besluiten of Podio nog steeds een leider is op gebied van low-code/no-code cloud-based management platformen of dat er een alternatief is die beter presteert. Dit wordt bepaald aan de hand van een lijst criteria die wordt opgesteld in de methodologie. Via een het uitbouwen van een uitgewerkte use case wordt een vergelijking gemaakt tussen Podio en de verschillende alternatieven.

\section{\IfLanguageName{dutch}{Opzet van deze bachelorproef}{Structure of this bachelor thesis}}%
\label{sec:opzet-bachelorproef} % TODO - OK

% Het is gebruikelijk aan het einde van de inleiding een overzicht te
% geven van de opbouw van de rest van de tekst. Deze sectie bevat al een aanzet
% die je kan aanvullen/aanpassen in functie van je eigen tekst.

De rest van deze bachelorproef is als volgt opgebouwd:

In Hoofdstuk~\ref{ch:stand-van-zaken} wordt een overzicht gegeven van de stand van zaken binnen het onderzoeksdomein, op basis van een literatuurstudie. Eerst en vooral wordt er toegelicht wat er precies verstaan wordt onder de term 'low-code development platform'. Vervolgens wordt meer informatie gegeven over Podio en de verschillende alternatieven die in deze studie aan bod komen.

In Hoofdstuk~\ref{ch:methodologie} wordt de methodologie toegelicht en worden de gebruikte onderzoekstechnieken besproken om een antwoord te kunnen formuleren op de onderzoeksvragen. Er wordt een use case uitgewerkt en daarna opgebouwd in Podio en zijn alternatieven. Op basis daarvan worden vergelijkingen gemaakt op gebied van werking, functionaliteiten, gebruikersvriendelijkheid, etc. om zo een conclusie en antwoord op de onderzoeksvraag te bekomen.

In Hoofdstuk~\ref{ch:conclusie}, tenslotte, wordt de conclusie gegeven en een antwoord geformuleerd op de onderzoeksvragen. Daarbij wordt ook een aanzet gegeven voor toekomstig onderzoek binnen dit domein.