%%=============================================================================
%% Conclusie
%%=============================================================================

\chapter{Conclusie}%
\label{ch:conclusie}

% TODO: Trek een duidelijke conclusie, in de vorm van een antwoord op de
% onderzoeksvra(a)g(en). Wat was jouw bijdrage aan het onderzoeksdomein en
% hoe biedt dit meerwaarde aan het vakgebied/doelgroep? 
% Reflecteer kritisch over het resultaat. In Engelse teksten wordt deze sectie
% ``Discussion'' genoemd. Had je deze uitkomst verwacht? Zijn er zaken die nog
% niet duidelijk zijn?
% Heeft het onderzoek geleid tot nieuwe vragen die uitnodigen tot verder 
%onderzoek?

Nu dat het onderzoek uitgevoerd is en de use case succesvol uitgebouwd is in elk platform, kan overgegaan worden tot de conclusie van deze vergelijkende studie. \\

Dit onderzoek biedt een meerwaarde voor Quivvy Solutions BV of andere bedrijven die zich specialiseren in Podio, omdat het weergeeft hoe andere low-code platformen die concurreren met Podio omgaan met eenzelfde use case en hoe dit mogelijks uitgewerkt kan worden. Het geeft ook inzicht in de implementatie van andere functionaliteiten zoals workflows en relaties in deze platformen. \\

Verder kan er besloten worden low-code/no-code development platformen in een enorme opmars zijn en dat ze, desondanks de vele uitdagingen, nog een grote impact zullen hebben op de toekomst van development. De mogelijkheid om als non-IT'er toch aan software-ontwikkeling te kunnen doen is duidelijk iets waar steeds meer en meer vraag naar is. \\

Vervolgens is uit het onderzoek gebleken dat AppSheet en Airtable een iets betere implementatie hebben voor het leggen van relaties tussen gegevens. Als het gaat om calculaties, blijft Podio nog steeds veruit het beste platform vanwege de flexibiliteit die het biedt. Podio geeft gebruikers namelijk veel vrijheid om complexe berekeningen en tekstverwerkingen te maken. Verder blinkt Airtable uit op gebied van performantie en gebruikerservaring. Het platform vereist het minste aantal klikken en de pagina's worden zeer snel ingeladen, waardoor gebruikers efficiënt kunnen werken zonder tijd te verspillen aan wachten. Wat betreft automatisaties kunnen de platformen als min of meer gelijkwaardig beschouwd worden, omdat ze bij elk platform zowel voordelen als nadelen hebben ten opzichte van de andere platformen. Hoewel de platformen verschillen in structuur, blijft de achterliggende werking grotendeels hetzelfde. Ze stellen gebruikers in staat om apps of databases te maken en te beheren, gegevens op te slaan en te organiseren, en aangepaste functionaliteiten toe te voegen aan hun workflows. Kortom, zowel Podio als zijn alternatieven zijn steeds sterke keuzes voor het bouwen van op maat gemaakte low-code software oplossingen. \\

Aangezien er steeds meer low-code platformen uitkomen op de markt is er zeker plaats voor een vervolgonderzoek waar Podio vergeleken kan worden met andere mogelijke alternatieven zoals Notion of Mendix. Daarnaast kan er ook dieper ingegaan worden op specifieke extensies of functionaliteiten binnen Podio. \\ 

Ten slotte kan als antwoord op de algemene onderzoeksvraag van deze studie, namelijk 'Met welke tools kan Quivvy Solutions hetzelfde bereiken als met Podio en hoe geïntegreerd werkt dit?', geconcludeerd worden dat het bedrijf zowel met Airtable als met Google AppSheet dezelfde doeleinden kan bereiken als met Podio, maar dat er wel een nodige inspanning moet gedaan worden om te leren werken met beide alternatieven. Daarnaast kan uit de resultaten van het onderzoek afgeleid worden dat uit de twee alternatieven, Airtable waarschijnlijk een betere keuze is voor het bedrijf dan AppSheet aangezien het bouwproces probleemloos verliep en het platform de beste gebruikerservaring heeft. \\ 