%---------- Inleiding ---------------------------------------------------------

\section{Introductie}%
\label{sec:introductie}

De afgelopen jaren hebben bedrijven steeds meer nood aan een gecentraliseerde structuur waarin al hun processen samenkomen en opgevolgd kunnen worden. Om die reden werd Citrix Podio ontwikkeld, een low-code tool waarmee dit soort cloud based business management platform opgebouwd kan worden. In de laatste jaren is de low-code/no-code markt enorm gestegen in populariteit. Volgens een rapport van Gartner, zou tegen 2024 zelfs meer dan 65\% van de applicaties gemaakt worden in een low-code/no-code omgeving \autocite{Costello2021}.

Quivvy Solutions BV is een bedrijf dat low-code software-oplossingen op maat maakt, met een minimum aan tools en een maximum aan functionaliteit. Hiervoor gebruiken zij Podio, maar ondertussen zijn er al een groot aantal concurrenten, zoals Google AppSheet en Airtable. Kunnen ze met deze tools hetzelfde bereiken als met Citrix Podio en hoe geïntegreerd werkt dit? In hoeverre is wat ze met Podio kunnen, ook mogelijk via AirTable en via het Google ecosysteem met Google App Sheets.

%---------- Stand van zaken ---------------------------------------------------

\section{State-of-the-art}%
\label{sec:state-of-the-art}

 Vorig academiejaar werd reeds een vergelijkende studie gemaakt tussen Podio, ClickUp en Airtable door \textcite{Spitaels2022}, dit onderzoek zal deze studie verderzetten. Podio zal vergeleken worden  AppSheet, een no-code development platform. Vervolgens zal op aanvraag van mijn stagebedrijf Airtable verder onderzocht worden, omdat het platform snel evolueert en al heel wat aanpassingen heeft aangebracht.
 
\subsection{Cloud based management}

Een cloud-based management platform wordt door \textcite{Spitaels2022} gedefiniëerd als 'Een tool die er voor gaat zorgen dat een bedrijf vanuit één centrale plek al zijn processen kan bijhouden, uitvoeren en bijsturen en dit terwijl iedereen met dezelfde data te werk gaat en ondertussen snel kan communiceren met elkaar.'
Podio was een voorloper in dit domein en is tegenwoordig één van de grootste spelers.

\subsection{Citrix Podio}

Citrix Podio is een cloud-based management platform dat gebruikers toelaat om op maat gemaakte business software te ontwikkelen. Het kan alle business processen aggregeren in een enkel platform en kan probleemloos uitgebreid worden naar de wensen van de gebruiker. Het voorziet diverse bouwblokken voor het bouwen van mobiele- en desktop applicaties en automatiseringen. Vervolgens bevat Podio een goede API waardoor het gemakkelijk externe software en tools kan integreren. Om die reden kan het eenvoudig externe applicaties onderhouden vanuit een enkel systeem \autocite{Quivvy}.  


\subsection{Airtable}

Airtable is een flexibele low-code/no-code tool waarmee op maat gemaakte applicaties gebouwd kunnen worden. Het is in staat om processen te visualiseren aan de hand van data uit een interne database. Daarnaast laat het toe om externe tools te integreren zodat taken zoals een email verzenden, kunnen geautomatiseerd worden. Daarbovenop heeft het ook ingebouwde functies waarmee het repetitieve taken kan automatiseren. Zo hoeft de gebruiker hieraan geen tijd te besteden en kan ze de focus leggen op zaken die belangrijker zijn \autocite{Airtable}.

\subsection{AppSheet}

AppSheet is een no-code development platform dat werd overgenomen door Google in 2020. Het laat toe om mobiele en desktop applicaties te bouwen zonder code te hoeven schrijven. Daarnaast is het in staat om bots aan te maken die taken zoals emails verzenden gaan automatiseren. Vervolgens is er ook de mogelijkheid om externe applicaties zoals SQL databases te integreren \autocite{AppSheet2020}. 


%---------- Methodologie ------------------------------------------------------
\section{Methodologie}%
\label{sec:methodologie}

Het onderzoek zal verlopen op basis van een lijst vergelijkingscriteria die door mijn stagebedrijf werd voorzien. 

\begin{itemize}
    \item Ten eerste zal nagegaan worden of het alternatief past bij het doel, namelijk low code business software die op maat gemaakt is voor diverse bedrijfscontexten. Hoeveel code is er nodig om dit te bereiken en hoe eenvoudig is het om alles vanuit één plaats te beheren?
    \item Ten tweede wordt onderzocht hoe flexibel het alternatief is. Zijn er eventuele beperkingen? Hoe gemakkelijk kunnen processen aangepast worden?
    \item Ten derde wordt gekeken in hoeverre de kernelementen van Podio aanwezig zijn in het alternatief. Is er een relationele database met de mogelijkheid om alles met elkaar te linken?  Zijn er calculatievelden om berekeningen uit te voeren of om dynamisch tekst te tonen? Is het alternatief in staat om automatiseringen uit te voeren. Kan het integreren met andere tools via API-connecties?
    \item Ten slotte wordt de gebruikerservaring onderzocht, hoe clean is de gebruikersinterface en hoe gemakkelijk is die te gebruiken? Hoeveel clicks, vensters, $\ldots$ zijn er om een bepaalde actie uit te voeren?
\end{itemize}

De vergelijkende studie zal antwoord bieden op volgende onderzoeksvragen:

\begin{itemize}
    \item Welke verschillen zijn er tussen Podio en zijn alternatieven?
    \item Op welke gebieden presteren de alternatieven beter of slechter dan Podio?
    \item Is de overstap van Podio naar een nieuwer alternatief de nodige inspanning waard?
\end{itemize}

%---------- Verwachte resultaten ----------------------------------------------
\section{Verwacht resultaat, conclusie}%
\label{sec:verwachte_resultaten}

Er wordt verwacht dat Podio goede resultaten zal behalen, omdat het een voorloper is en een zeer groot aanbod aan features heeft. Vervolgens wordt er verwacht dat Airtable minstens even sterk zal presteren als Podio. In het onderzoek van \textcite{Spitaels2022} werd geconcludeerd dat het een gelijkwaardig platform is aan Podio op gebied van functionaliteiten, maar dat het wel slechter presteerde op gebied van gebruikerservaring. Om die reden wordt er verwacht dat Airtable enige inspanningen gedaan heeft om zijn gebruikerservaring te verbeteren. Ten laatste wordt er verwacht dat AppSheet ook hoge resultaten zal halen, omdat het eigendom is van gigant Google, die een groot aantal resources bevat.

De resultaten van het onderzoek kunnen een meerwaarde vormen voor mijn stagebedrijf. Ze werken dagelijks met Podio en willen steeds op de hoogte zijn van andere opkomende tools.

