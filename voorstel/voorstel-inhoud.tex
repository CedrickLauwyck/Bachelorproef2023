%---------- Inleiding ---------------------------------------------------------

\section{Introductie}%
\label{sec:introductie}

% TODO Zeker toevoegen:
% TODO - concrete probleemsituatie OK
% TODO - doelstelling OK
% TODO - centrale onderzoeksvraag OK
% TODO - deelvraag OK

De afgelopen jaren hebben bedrijven steeds meer nood aan een digitale gecentraliseerde structuur waarin al hun processen en gegevens samenkomen. Hieruit kunnen ze hun processen beter opvolgen en vervolgens betere business beslissingen maken \autocite{CFI2022}. Om die reden werd Podio \footnote{https://www.podio.com/} ontwikkeld, een low-code tool waarmee dit soort cloud based business management platform opgebouwd kan worden. In de laatste jaren is de low-code/no-code markt enorm gegroeid. Volgens een rapport van \textcite{Costello2021}, zou tegen 2024 zelfs meer dan 65\% van de applicaties gemaakt worden in een low-code/no-code omgeving \autocite{Brown2022}.
% TODO bronvermelding: https://cyclr.com/blog/low-code-is-revolutionising-the-software-industry OK

% De reden voor het zoeken naar Podio-alternatieven is tweevoudig: enerzijds inderdaad zien wat de ontwikkelingen zijn en of Podio daartegen standhoudt, anderzijds kijken of we een goed alternatief hebben mochten we niet meer met Podio willen werken. Podio heeft in de laatste jaren wat stabiliteitsproblemen gehad. Er is het een en ander aan het verbeteren, maar we houden daardoor de optie open om op termijn met een ander platform te werken (maar dan moet het wel een echt goed alternatief zijn). 

Quivvy Solutions BV \footnote{https://quivvy.com/} is een bedrijf dat low-code software-oplossingen op maat maakt, met een minimum aan tools en een maximum aan functionaliteit, hiervoor gebruiken zij Podio. Alhoewel de tool voldoet aan hun eisen, hebben ze de laatste jaren al een aantal stabiliteitsproblemen ondervonden. Daarnaast zijn er ook al een groot aantal concurrerende tools op de markt gekomen. Om deze redenen houden ze de optie open om op termijn over te schakelen naar een goed alternatief. Dit onderzoek biedt dus een antwoord op volgende vragen:

\begin{itemize}
    \item Met welke tools kan Quivvy Solutions hetzelfde bereiken als met Podio en hoe geïntegreerd werkt dit? 
    \item Welke verschillen zijn er tussen Podio en zijn alternatieven?
    \item Op welke gebieden presteren de alternatieven beter of slechter dan Podio?
    \item Is de overstap van Podio naar een nieuwer alternatief de nodige inspanning waard?
\end{itemize}


%---------- Stand van zaken ---------------------------------------------------

\section{State-of-the-art}%
\label{sec:state-of-the-art}

% TODO Analyse van de huidige processen van bedrijf en waarvoor Podio wordt ingezet OK
% TODO Ga ook in op wat low-code business management NOK
% TODO Literatuurstudie te veel op de vlakte en ontbreekt expliciete link met probleemstelling OK

% TODO Resultaten spitaels inzetten OK
In 2022 werd reeds een vergelijkende studie gemaakt tussen Podio, ClickUp en Airtable door \textcite{Spitaels2022}. Uit dit onderzoek werd besloten dat Podio en Airtable gelijkwaardig zijn, maar ook dat Podio de beste gebruikerservaring voorziet. Dit onderzoek zal deze studie verderzetten. Zoals eerder vermeld zijn er een groot aantal alternatieven die onderzocht kunnen worden \ref{table:Tabel 1}. In samenspraak met Quivvy Solutions worden in deze studie volgende alternatieven onderzocht:

% TODO redenen teovoegen OK
\begin{itemize}
    \item Google AppSheet\footnote{https://about.appsheet.com/home/}; naast het feit dat deze tool over de basisfunctionaliteiten van Podio beschikt, is deze tool eigendom van Google.
     Daarom is het interessant om te onderzoeken hoe het samenspeelt met de bredere Google suite om zo op zijn geheel een mogelijks beter alternatief te vormen.
    \item Airtable\footnote{https://www.airtable.com/}; deze tool werd in 2022 onderzocht voor Quivvy Solutions en wordt sindsdien ook door hen gebruikt. Omdat dit platform snel evolueert en al een groot aantal aanpassingen heeft aangebracht sinds het vorige onderzoek, willen ze dat het verder onderzocht wordt.
\end{itemize}

\begin{table}[ht]
    \centering
    % TODO Add bron: https://analyticsindiamag.com/10-low-code-no-code-platforms-every-developer-should-know-of/ OK
    \caption{\label{tab:Tabel 1} Lijst met Podio alternatieven die onderzocht kunnen worden \autocite{Tasmia2022}.}
    \begin{tabular}{ | c | }
        \hline
        \textbf{Low-code/no-code platformen} \\
        \hline
        Appian \\
        Mendix \\
        Nintex \\
        Visual LANSA \\
        Quixy \\
        Airtable \\
        Caspio \\
        Kissflow \\
        Quickbase \\
        ZohoCreator \\
        \hline
    \end{tabular}

    {\raggedright \textit{Opm. Er zijn nog alternatieven die niet in deze lijst vermeld staan.} \par}
\end{table}
 
 % TODO Iets verder uitbreiden OK
\subsection{Cloud based management}

Een cloud-based management platform wordt door \textcite{Spitaels2022} gedefiniëerd als 'Een tool die er voor gaat zorgen dat een bedrijf vanuit één centrale plek al zijn processen kan bijhouden, uitvoeren en bijsturen en dit terwijl iedereen met dezelfde data te werk gaat en ondertussen snel kan communiceren met elkaar.' 

% TODO info toevoegen OK
\subsection{Low-code platform}

Een low-code platform wordt door \textcite{Waszkowski2019} omschreven als 'Een set van tools voor programmeurs en niet-programmeurs'. In dit type development platform wordt gesteund op een grafische user interface (GUI) voor het ontwerpen en bouwen van applicaties. Het stelt de gebruiker in staat om snel en met weinig programmeerkennis, een volledige business applicatie op te bouwen \autocite{Waszkowski2019}. De meest voorkomende kenmerken van een low-code platform zijn een GUI ontwerper met voorgemaakte widgets of palettes en de functie om externe datasources te koppelen. Bovendien bevat het systeem vaak een bibliotheek van standaardoperaties, zoals bijvoorbeeld wiskundige berekeningen en de mogelijkheid om externe functies uit te voeren via een API \autocite{Bock2021}.

% TODO Link tussen Podio & Quivvy, waavoor wordt het gebruikt en hoe wordt het ingezet? OK
% TODO BRON https://quivvy.com/nl/software OK
\subsection{Podio}

Podio is een cloud-based management platform dat gebruikers toelaat om op maat gemaakte business software te ontwikkelen. Het kan alle business processen aggregeren in een enkel platform en kan probleemloos uitgebreid worden naar de wensen van de gebruiker. Het voorziet diverse bouwblokken voor het bouwen van mobiele- en desktop applicaties en automatiseringen. Vervolgens bevat Podio een goede API waardoor het gemakkelijk externe software en tools kan integreren \autocite{QuivvyPodio}. Het doel van Podio is om alle gegevens en communicatie te omvatten in één tool die overal gebruikt kan worden \autocite{Podio}.  

Quivvy Solutions BV onderzoekt de business processen van hun klanten en welke tools zij daarvoor gebruiken. Hierna gaan ze deze processen gaan configureren in Podio. Daarbovenop onderzoeken ze ook welke tools van de klant gelinkt, of zelfs vervangen kunnen worden door Podio. Verder gaan ze afzonderlijke processen omzetten naar workflows en die, indien mogelijk, ook automatiseren \autocite{QuivvySoftware}. Kortom analyseert Quivvy Solutions de processen en gegevens van hun klanten en zetten deze vervolgens om naar een Podio-omgeving waarmee de klant in staat is om zijn volledige bedrijfsinfrastructuur te beheren.

% TODO Waarom is dit een goed alternatief? OK
\subsection{Airtable}

Airtable is een flexibele low-code/no-code tool waarmee op maat gemaakte applicaties gebouwd kunnen worden. Het is in staat om processen te visualiseren aan de hand van data uit een interne database. Daarnaast laat het toe om externe tools te integreren zodat taken zoals een email verzenden, kunnen geautomatiseerd worden. Daarbovenop heeft het ook ingebouwde functies waarmee het repetitieve taken kan automatiseren. Zo hoeft de gebruiker hieraan geen tijd te besteden en kan ze de focus leggen op zaken die belangrijker zijn \autocite{Airtable}.

De tool beschikt over functionaliteiten die gelijkaardig zijn aan die van Podio, zoals automaties, integraties en views. Daarbovenop heeft het ook unieke features zoals een Interface Designer \autocite{Airtable}. Hierdoor kan Airtable dus mogelijks een goed alternatief vormen voor Podio.

% TODO Waarom is dit een goed alternatief? OK
\subsection{AppSheet}

AppSheet is een no-code development platform dat werd overgenomen door Google in 2020. Het laat toe om mobiele en desktop applicaties te bouwen zonder code te hoeven schrijven. Daarnaast is het in staat om bots aan te maken die taken zoals emails verzenden gaan automatiseren. Vervolgens is er ook de mogelijkheid om externe applicaties zoals SQL databases te integreren \autocite{AppSheet2020}. 

Net zoals Airtable beschikt AppSheet ook de basisfunctionaliteiten van Podio \autocite{AppSheet2020}. Bovendien is de tool eigendom van Google, waardoor het toegang heeft tot grote omvang van resources waarover het bedrijf beschikt. Dit maakt het platform een interessante keuze om te onderzoeken.

%---------- Methodologie ------------------------------------------------------
\section{Methodologie}%
\label{sec:methodologie}

% TODO Proof of Concept OK
% TODO Technologische component OK
% Het persoonlijke luik zou inderdaad een use case zijn uit onze klantenprojecten, en zien of je deze gemakkelijk kan nabouwen en of alles even vlot werkt.

\subsection{Proof of Concept}

Als Proof of Concept wordt een willekeurige use case genomen uit de klantenprojecten van Quivvy Solutions. Deze case wordt dan nagebouwd in de verschillende alternatieve platformen en geëvalueerd op basis van een lijst vergelijkingscriteria. 

\subsection{Vergelijkingscriteria}

\begin{itemize}
    \item Ten eerste zal nagegaan worden of het alternatief past bij het doel, namelijk low code business software die op maat gemaakt is voor diverse bedrijfscontexten. Hoeveel code is er nodig om dit te bereiken en hoe eenvoudig is het om alles vanuit één plaats te beheren?
    \item Ten tweede wordt onderzocht hoe flexibel het alternatief is. Zijn er eventuele beperkingen? Hoe gemakkelijk kunnen processen aangepast worden?
    \item Ten derde wordt gekeken in hoeverre de kernelementen van Podio aanwezig zijn in het alternatief. Is er een relationele database met de mogelijkheid om alles met elkaar te linken?  Zijn er calculatievelden om berekeningen uit te voeren of om dynamisch tekst te tonen? Is het alternatief in staat om automatiseringen uit te voeren? Kan het integreren met andere tools via API-connecties?
    \item Ten slotte wordt de gebruikerservaring onderzocht, hoe gemakkelijk is de gebruikersinterface te gebruiken; Hoeveel clicks, vensters, $\ldots$ zijn er om een bepaalde actie uit te voeren?
\end{itemize}

De vergelijkende studie zal antwoord bieden op volgende onderzoeksvragen:

\begin{itemize}
    \item Welke verschillen zijn er tussen Podio en zijn alternatieven?
    \item Op welke gebieden presteren de alternatieven beter of slechter dan Podio?
    \item Is de overstap van Podio naar een nieuwer alternatief de nodige inspanning waard?
\end{itemize}

%---------- Verwachte resultaten ----------------------------------------------
\section{Verwacht resultaat, conclusie}%
\label{sec:verwachte_resultaten}

Er wordt verwacht dat Podio goede resultaten zal behalen, omdat het een voorloper is en een zeer groot aanbod aan features heeft. Vervolgens wordt er verwacht dat Airtable minstens even sterk zal presteren als Podio. In het onderzoek van \textcite{Spitaels2022} werd geconcludeerd dat het een gelijkwaardig platform is aan Podio op gebied van functionaliteiten, maar dat het wel slechter presteerde op gebied van gebruikerservaring. Om die reden wordt er verwacht dat Airtable enige inspanningen gedaan heeft om zijn gebruikerservaring te verbeteren. Ten laatste wordt er verwacht dat AppSheet ook hoge resultaten zal halen, omdat het eigendom is van gigant Google en gebruik kan maken van Google's grote aantal resources.

De resultaten van het onderzoek kunnen een meerwaarde vormen voor Quivvy Solutions. Ze werken dagelijks met Podio en willen steeds op de hoogte zijn van andere opkomende tools. Bovendien zijn ze ook bereid om over te stappen naar alternatief indien dit platform beter zou zijn.

